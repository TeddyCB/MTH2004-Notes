%! Author = teddy
%! Date = 18/01/2022

% Preamble
\documentclass[11pt]{article}
\title{MTH2004 - Vector Calculus and Applications: Scalar and Vector Fields}

% Packages
\usepackage{amsmath}
\usepackage{graphicx}
\usepackage{pgfplots}
\usepackage{tikz}
\usetikzlibrary{calc}

\pgfplotsset{compat= newest}
% Document
\begin{document}
    \maketitle
    \section{Definitions}\label{sec:definitions}
    Many physical quantities have values at every different point in a particular
    region of space.
    For example:\\
    \newline
    a) The temperature in a room.\\
    b) Gravitational acceleration.\\
    c) Velocity of water flow.
    \\
    \\
    The term \textbf{field} is used to mean both the region of space and the value of the physical
    quantity in that region:\\\\
    \textbullet \; For a scalar quantity: a \textbf{scalar field} $\phi(\vec{r}) = \phi(x,y,z)$\\
    \textbullet \; For a vector quantity: a \textbf{vector field} $\vec{F}(\vec{r}) = \vec{F}(x,y,z)$\\
    \subsection{Level Curves and Level surfaces}\label{subsec:level-curves-and-level-surfaces}
    The gravitational potential of Earth is a scalar field and near the surface can be approximated as: $\phi(z) = gz$.
    Where an arbitrary height z = 0 has been chosen as the reference level.
    The potential field is related to the gravitational potential energy $U$, between a mass $m$ and the Earth
    as $U = mgz = m\phi$.\\\\
    Suppose on a hill we draw a curve corresponding to a constant value of $\phi(\vec{r}) = C$.
    This curve is called a \textbf{level curve} of $\phi$.\\\\
    These level curves correspond to the contour lines on an ordinance survey map that indicate height.
    \begin{figure}
        \begin{center}
            \label{fig:figure}
            \includegraphics[scale = 1]{survey_map}
            \caption{Graph with Contour Lines}
        \end{center}
    \end{figure}
    \textbf{Level surfaces} of a scalar field are surfaces where all points share the same value of the scalar field,
    $\phi(\vec{r}) = C$.\\
    \paragraph{Example of Level Curves}\\
    Consider the function$f(x,y) = x^2 + y^2$, the level curves are $x^2 + y^2 = C$ are centric circles of radius $\sqrt {C}$.
    \begin{figure}[h]
        \label{fig:figure2}
        \begin{center}
            \includegraphics[scale = .70]{level_curve_eg}
            \caption{Level Curve example where $C = 16$}
        \end{center}
    \end{figure}
    \paragraph{Example 2: Level Surfaces}
    For $f(x,y,z) = x^2 + y^2 + z^2$ the level surfaces are $ x^2 + y^2 + z^2 = C$ are are concentric spheres
    of radius $\sqrt {C}$.
    \begin{figure}[h]
        \label{fig:figure3}
        \begin{center}
            \includegraphics[scale = .70]{level_surface_eg}
            \caption{Level Surface example where $C = 16$}
        \end{center}
    \end{figure}
    \subsection{Vector Fields and Field Lines}\label{subsec:vector-fields-and-field-lines}
    Vector fields in two dimensions can be visualised by drawing the vector at a sequence of points or on a grid,
    with \textbf{the length and direction of the arrow denoting the magnitude and direction of the vector} respectively.
    \\
    A \textbf{field line} is a curve whose tangent is parallel to the vector field at each point along the curve.
    With respect to fluid dynamics, field lines are known as \textbf{streamlines} and show the direction in which
    fluid particles travel.\\
    The \textbf{density of field liens} is an indication of the magnitude of the vector field.\\

    \section{Differentiating Scalar Fields}\label{sec:differentiating-scalar-fields}
    We define the \textbf{gradient} of a scalar field $f(x,y,z)$ as the vector\\:
    \begin{equation}
        \label{eq:equation}
        grad \; f = \nabla f = (\frac{ \partial f}{\partial x},\: \frac{ \partial f}{\partial y},\: \frac{ \partial f}{\partial z})\label{eq:equation26}
    \end{equation}
    where $\nabla$ is a differential operator:
    \begin{equation}
        grad  = \nabla  = (\frac{ \partial }{\partial x},\: \frac{ \partial }{\partial y},\: \frac{ \partial }{\partial z}\;)
        = \hat{x} \frac{\partial}{\partial x} + \hat{y} \frac{\partial}{\partial y} + \hat{z} \frac{\partial}{\partial z}
        \label{eq:equation2}
    \end{equation}
    \textbf{NB:} despite $f$ being a scalar field, $\nabla f$ is a vector field!
    \\
    \paragraph{$\nabla f$ with respect to level surfaces}
    The gradient $\nabla f$ is orientated perpendicular to the level surfaces $f(\vec{r}) = C$, in the direction of the steepest ascent,
    with magnitude equal to the rate of change in that direction.
    Mathematically:
    \begin{equation}
        \label{eq:equation3}
        | \nabla f | = \frac{df}{ds}
    \end{equation}
    Where $s$ is the distance measured along the surface normal vector (parallel to $\nabla f$) $\hat{n}$.

    \paragraph{The properties of the gradient} For any scalar functions of position $f(x,y,z)$ and $g(x,y,z)$ and any constant
    $c$:
    \begin{subequations}
        \begin{align}
            \nabla (f + g) = \nabla f + \nabla g\\
            \nabla (cf) = c \nabla f\\
            \nabla (fg) = f \nabla g + g \nabla f \: \text{[product rule]}\\
            \nabla f(g) = f'(g) \nabla g  \: \text{[chain rule]}\\
        \end{align}
    \end{subequations}

    \subsection{The directional derivative}\label{subsec:the-directional-derivative}
    Previously we considered the rate of change of a scalar field in the direction normal to its level surfaces.
    But we may calculate the rate of change in any arbitrary direction.\\
    Consider a direction, defined as the unit vector $\hat{a}$, and a displacement vector $d\vec{r} = \hat{a} ds$, where
    $ds$ is the distance along $\vec{a}$.
    Then:
    \begin{equation}
        \label{eq:equation4}
        df = \nabla f \cdot \hat{a} \; ds \Rightarrow \frac{df}{ds} = \nabla f \cdot \hat{a} = (\hat{a} \cdot \nabla) f
    \end{equation}
    This is known as the \textbf{directional derivative} of $f$ in the direction of $\hat{a}$.
    We can also write it as:
    \begin{equation}
        \label{eq:equation5}
        \frac{df}{ds} = | \nabla f | \cos (\theta)
    \end{equation}
    where $\theta$ is the angle between $\nabla f$ and $\hat{a}$.\\
    \textbf{NB:} Since $-1 \leq \cos \theta \leq 1$:\\
    \textbullet $f$ increases most rapidly in the direction of $\nabla f$\\
    \textbullet $f$ decreases most rapidly in the direction of $-\nabla f$

    \subsection{Vector functions}\label{subsec:vector-functions}
    \subsubsection{Parameterised curves}
    A curve $C$ in three-dimensional space is an inherently on-dimensional object.
    The position vector of any point along the curve may be characterised using one parameter:
    \begin{equation}
        \label{eq:equation6}
        \vec{r}(t) = x(t)\hat{x} + y(t)\hat{y} + z(t)\hat{z}
    \end{equation}
    \subsubsection{Tangent vector and tangent lines}
    The derivative of the position vector $\vec{r}(t)$ w.r.t the parameter $t$ is the \textbf{tangent vector}:
    \begin{equation}
        \label{eq:equation7}
        \vec{r'}(t) = \frac{dx}{dt}\hat{x} + \frac{dy}{dt}\hat{y} + \frac{dz}{dt}\hat{z}

    \end{equation}
    The \textbf{tangent line} to the curve at point P with position vector $\vec{r}(t_0)$ is the line
    through the point P and parallel with $\vec{r'}(t_0)$:
    \begin{equation}
        \label{eq:equation8}
        \vec{q}(w) = \vec{t_0} + w\vec{r'}(t_0)
    \end{equation}
    \subsubsection{Length of a curve}
    \paragraph{Planar curve}
    The length L of a curve C given in the form $y= F(x)$, $a \leq x \leq b$ is:
    \begin{equation}
        \label{eq:equation9}
        L = \int_{a}^{b} \sqrt {1 + \left( \frac{dy}{dx}\right)^2} \: dx
    \end{equation}
    Suppose that C can be described by the parametric $\vec{r}(t) = x(t)\hat{x} + y(t)\hat{y}$, $t \in [\alpha, \beta]$,
    where $\frac{dx}{dt} = x'(t) > 0$.
    This means that C is traversed once, from left to right as $t$ increases from $\alpha$ to $\beta$ and $x(\beta) = b$.\\
    If $\frac{dx}{dt} = x'(t) \neq 0$, the slope $\frac{dy}{dx}$ of the tangent to the curve can be found via the chain rule:
    \begin{subequations}
        \begin{center}
            \begin{align}
                \frac{dy}{dt} = \frac{dy}{dx} \cdot \frac{dx}{dt}
                \Rightarrow \frac{dy}{dx} = \frac{y'(t)}{x'(t)}
            \end{align}
        \end{center}
        If we think of the curve as being traced by a moving particle, then $x'(t)$ and $y'(t)$ are the horizontal
        and vertical velocities of the particle, respectively, and the expression above says that the slope of the
        tangent is the ratio of these velocities.\\
        Using the substitution rule, we obtain:
        \begin{center}
            \begin{align}
                L = \int_{\alpha}^{\beta} \sqrt {1 + \left( \frac{\frac{dy}{dt}}{\frac{dx}{dt}} \right)^2} \frac{dx}{dt}\: dt
            \end{align}
        \end{center}
        since $\frac{dx}{dt} = x'(t) > 0$, we have:
        \begin{center}
            \begin{align}
                L = \int_{\alpha}^{\beta} \sqrt {\left( \frac{dx}{dt} \right)^2 + \left( \frac{dy}{dt} \right)^2}\: dt
            \end{align}
        \end{center}
    \end{subequations}
    \paragraph{Generalisation}
    The length of a smooth curve $\vec{r}(t) = x(t)\hat{x} + y(t)\hat{y} + z(t)\hat{z}$ for $a \leq t \leq b$, is:
    \begin{equation}
        \label{eq:equation10}
        \ell = \int_{a}^{b} \left| \frac{d\vec{r}}{dt} \right| dt = \int_{a}^{b} | \vec{r'}(t) | \:dt =
        \int_{a}^{b} \sqrt {\vec{r'}(t) \cdot \vec{r'}(t)} \:dt
    \end{equation}
    If we replace the fixed value $b$ by the parameter $t$, we obtain the arc length as a function of $t$:
    \begin{equation}
        \label{eq:equation11}
        s(t) = \int_{t_0}^{t} |\vec{r'}(\tau)| d\tau
    \end{equation}
    which is the directed distance between points $P(\vec{r}(t_0))$ to $P(\vec{r}(t))$ along the curve.\\
    Further, if we differentiate and square the arc length we obtain:
    \begin{equation}
        \label{eq:equation12}
        (ds)^2 = d\vec{r} \cdot d\vec{r} = (dx)^2 + (dy)^2 + (dz)^2
    \end{equation}
    Where $ds$ is called the \textbf{line element} of the curve C .
\end{document}