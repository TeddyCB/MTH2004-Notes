%! Author = teddy
%! Date = 28/02/2022

% Preamble
\documentclass[11pt]{article}
\title{Application: Fluid Dynamics}
% Packages
\usepackage{amsmath}
\usepackage{esint}
% Document
\begin{document}
\maketitle
    \section{The Continuity Equation}\label{sec:the-continuity-equation}
        \subsection{Preliminaries}\label{subsec:preliminaries}
            We consider the motion of a fluid in a region $R$ having no sources or sinks in $R$, meaning 
            no points at which fluid is produced or disappears.\\
            \begin{enumerate}
                \item \textbf{In a fluid state}, a volume element of fluid has:\begin{enumerate}
                          \item \textbf{mass density} $\rho(\vec{r}, t)$, \textbf{i.e} a mass per unit volume that
                          can be a function of spatial location and time.
                          \begin{enumerate}
                              \item If the density is uniform in space and constant with time, the fluid is called \textbf{incompressible} with $\rho = \rho_0$.
                              \item Otherwise, it is \textbf{compressible}.
                          \end{enumerate}
                          \item \textbf{pressure} $p(\vec{r},t)$, \textbf{i.e} force exerted by the fluid per unit area that can be a function of spatial location and time.
                          \item \textbf{fluid velocity} $\vec{u}(\vec{r},t)$, \textbf{i.e} velocity of the fluid volume element at location $\vec{r}$ at time $t$.
                                                                                   \begin{enumerate}
                                                                                       \item If the velocity is constant with time, the fluid flow is called \textbf{steady}, with $\vec{u}(\vec{r})$ only:
                                                                                                \begin{subequations}
                                                                                                    \label{eq:equation}
                                                                                                    \begin{align*}
                                                                                                        \frac{\partial \vec{u}}{\partial t} &= 0\\
                                                                                                        \frac{\partial \rho}{\partial t} &= 0\\
                                                                                                        \frac{\partial p}{\partial t} &= 0
                                                                                                    \end{align*}
                                                                                                \end{subequations}
                                                                                       \item Otherwise, it is \textbf{unsteady}.
                                                                                       Some stagnation point $\vec{r}*$ may exist where $\vec{u}(\vec{r}*,t) = 0$
                                                                                   \end{enumerate}
                \end{enumerate}
                \item The concept of the fluid state covers both liquids, gases and plasmas.
                        \begin{enumerate}
                            \item \underline{Liquids:} Generally incompressible, \textbf{i.e} fluid density uniform in space and constant in time:$\rho = \rho_0$
                            \item \underline{Gases:} Generally compressible, \textbf{i.e.} density depends on location in space and may vary with time:$\rho = \rho(\vec{r},t)$
                            \item \underline{Plasmas: } Generally compressible, combination of fluids of positive and negative charges that is quasi-neutral on the fluid scale
                        \end{enumerate}
                \item Two approaches in considering a volume element of fluid:
                    \begin{enumerate}
                        \item \textbf{Eulerian approach: } We consider the fluid flow at a point in space as time progresses.
                        This implies that in the volume element at that location the identity of the material continuously changes as fluid flows away and into that
                        volume.\\
                        This approach is comparable to observing the fluid flow of a stream from a fixed point on the embankment
                        \item \textbf{Lagrangian approach: }We consider the fluid flow as we track the position of the volume element in space with time.
                        The identity of teh fluid material in the volume element is preserved. \\
                        This approach is comparable to observing the fluid flow of a stream whilst moving with the flow in a boat.
                    \end{enumerate}
                \item Let $f(x,y,z,t)$ denote some quantity of interest in the fluid motion, \textbf{e.g.} a component of velocity $\vec{u}$ or mass density $\rho$.\\\\
                    \begin{subequations}
                        $\frac{\partial f}{\partial t}$: Eulerian rate of change of $f$ at a fixed position in space $(x,y,z)$\\\\
                        $\frac{df}{dt}$: Lagrangian rate of change of $f$ following the fluid \textbf{i.e.} $x(t), y(t), z(t)$ change
                        with time at the local flow velocity:
                        \begin{align*}
                            \frac{dx}{dt} = u_x && \frac{dy}{dt} = u_y && \frac{dz}{dt} = u_z
                        \end{align*}
                        We can connect the Lagrangian and Eulerian rates of change:
                        \begin{align*}
                            \frac{Df}{Dt} = \frac{\partial f}{\partial t} + (\vec{u} \cdot \nabla)f
                        \end{align*}
                    \end{subequations}
            \end{enumerate}
        \subsection{Mass flux through a surface}\label{subsec:mass-flux-through-a-surface}
            Consider a fluid with mass density $\rho$ and velocity $\vec{u}$.\\
            The flux of the vector field $\rho \vec{u}$ through a surface element $d\vec{S}$ is called the \textbf{mass flux}
            \begin{equation}
                \label{eq:equation2}
                \rho \vec{u} \cdot d\vec{S} = \rho \vec{u} \cdot \hat{n} dS
            \end{equation}
            Consider a closed surface $S$ bounding an arbitrary spatially \underline{fixed} volume $V$ of fluid.
            The \textbf{total mass flux} through the whole surface:
            \begin{equation}
                \label{eq:equation3}
                \oiint_{S} \rho \vec{u} \cdot d\vec{S}
            \end{equation}
            where we have the convention that hte surface normal points outwards.
        \subsection{Conservation of mass}\label{subsec:conservation-of-mass}
            At any time the mass of the fluid within a fixed volume $V$:
            \begin{subequations}
                \begin{align*}
                    M = \iiint_{V} \rho dV
                \end{align*}
                If the mass changes in time from $M(t)$ to $M(t + dt) = M(t) dM(t)$ due to the fluid leaving or
                entering the volume, then the rate of change of mass with time is:
                \begin{align*}
                    \frac{dM}{dt} = \frac{d}{dt} \iiint_V \rho dV = \iiint_V \frac{\partial \rho}{\partial t} dV
                \end{align*}
                This rate must be balanced by a mass flux through the bounding surface:
                \begin{align*}
                    \frac{dM}{dt} = -\oiint_S \rho \vec{u} \cdot d\vec{S}
                \end{align*}
                The minus sign is added because when fluid flows outwards through the surface, the mass flux is positive, but it corresponds
                to a decrease in mass inside the volume.
                We obtain:
                \begin{align*}
                    \oiint_S \rho \vec{u} \cdot d\vec{S} = \iiint_V \nabla \cdot (\rho \vec{u}) dV
                \end{align*}
                Mass conservation then becomes:
                \begin{align*}
                    \iiint_V \frac{\partial \rho}{\partial t} dV = - \iiint_V \nabla \cdot (\rho \vec{u}) dV
                \end{align*}
                Finally, because the volume $V$ is chosen arbitrarily, we can remove the volume integrals to find the vecotr form
                of mass conservation:
                \begin{align*}
                    \frac{\partial \rho}{\partial t} + \nabla \cdot (\rho \vec{u}) = 0
                \end{align*}
                This equation is called the \textbf{Continuity Equation}, or \textbf{mass conservation equation}.
                It describes at a location that rate of change of density over time is due to an expanding (or contracting) flow
                \\We can write the Lagrangian version of the Continuity Equation as:
                \begin{align*}
                    \frac{D \rho}{D t} + \rho \nabla \cdot \vec{u} = 0
                \end{align*}
                \begin{enumerate}
                    \item \textbf{If the fluid is \underline{incompressible}:}
                        \begin{align*}
                            \frac{D \rho}{D t} = \left[ \frac{\partial}{ \partial t} + \vec{u} \cdot \nabla \right] \rho = 0
                        \end{align*}
                        \begin{enumerate}
                            \item Fluid of constant and uniform density
                            \item Or, if the fluid is not uniform, the shape is advected (without changing shape) in the direction of $\vec{u}$,
                            \textbf{i.e} any inhomogeneity is transported with the flow:
                                \begin{align*}
                                    \text{incompressible} \Leftrightarrow \nabla \cdot \vec{u} = 0
                                \end{align*}
                            \textbf{Physical Interpretation:} If a fluid is incompressible, it cannot be expanded or compressed at any location or at any time.
                        \end{enumerate}
                    \item \textbf{if the flow is \underline{steady}:}
                        the fluid is constant in time \textbf{i.e} $\frac{\partial \rho}{\partial t} = 0$
                        \begin{align*}
                            \text{steady flow} \Leftrightarrow \nabla \cdot (\rho \vec{u}) = 0
                        \end{align*}
                        \textbf{Physical meaning:} The mass flux is conserved, \textbf{i.e} the same amount of mass enters a volume as it leaves
                \end{enumerate}
            \end{subequations}
        \subsection{Particle paths or pathlines}\label{subsec:particle-paths-or-pathlines}
            A particle path or pathline is the trajectroy of a fluid element of fixed identity as a function of time $\vec{r}(\vec{r}_0, t)$,
            where $\vec{r}_0$ is the initial position of the fluid element.
            At the position $\vec{r}$ at time $t$ the fluid element moves with the velocity $\vec{u}(\vec{r},t)$.
            Therefore, we can repeat the Lagrangian rate of change:
            \begin{subequations}
                \begin{align*}
                    & \frac{d\vec{r}}{dt} = \vec{u}(\vec{r}, t)\\
                    & \vec{r}(t) = x(t)\hat{x} + y(t)\hat{y} + z(t)\hat{z}
                \end{align*}
                or in component form:
                \begin{align*}
                    \frac{dx}{dt} = u_x && \frac{dy}{dt} = u_y && \frac{dz}{dt} = u_z
                \end{align*}
            \end{subequations}
    \subsection{Equation of motion}\label{subsec:equation-of-motion}
        \subsection{Acceleration}\label{subsec:acceleration}
            Applying the Lagrangian derivative to each velocity component $(u_x, u_y, u_z)$ gives the acceleration
            of a moving fluid element:
            \begin{subequations}
                \begin{align*}
                    \vec{a}(\vec{r},t) = \frac{D\vec{u}}{Dt} = \frac{\partial \vec{u}}{\partial t} + (\vec{u} \cdot \nabla) \vec{u}
                \end{align*}
                where $ ( \vec{u} \cdot \nabla) \vec{u}$ is called the \textbf{advective acceleration}
                \begin{enumerate}
                    \item For a \textbf{\underline{steady flow}}: $\frac{\partial \vec{u}}{\partial t} = 0$ implies:
                        \begin{align*}
                            \vec{a} = (\vec{u} \cdot \nabla ) \vec{u}
                        \end{align*}
                    \item We define \textbf{vorticity} of a flow field as:
                        \begin{align*}
                            \vec{\zeta} = \nabla \times \vec{u}
                        \end{align*}
                    The vorticity is orthogonal to the velocity field.
                    It has physical units of $s^{-1}$.
                    A flow with zero vorticity \textbf{i.e} $\nabla \times \vec{u} = \vec{0}$ is called an \textbf{irrotational flow}
                    The acceleration can be rewritten as::
                        \begin{align*}
                            \vec{a} = \frac{\partial \vec{u}}{\partial t} + \frac{1}{2} \nabla | \vec{u} |^2 - \vec{u} \times \vec{\zeta}
                        \end{align*}
                \end{enumerate}  
            \end{subequations}
        \subsection{Pressure Force}\label{subsec:pressure-force}
            A fluid in a volume element $dV$ exerts a force on its surrounding fluid \textbf{i.e} $d\vec{F} = p \hat{n} dS$,
            where $p$ s the pressure at any point on the bounding surface $S$.
            Conversely, the surrounding fluid exerts a pressure on the fluid volume:
            \begin{subequations}
                \begin{align*}
                    d\vec{F} = -p \hat{n} dS
                \end{align*}
                The total net force exerted on the fluid volume is:
                \begin{align*}
                    \vec{F} = - \oiint_S p \hat{n} dS
                \end{align*}
                Introduce an arbitrary uniform vector field $\vec{a}$ to write the scalar product:
                \begin{align*}
                    - \vec{a} \cdot \oiint_S p \hat{n} dS &= - \oiint_S p \vec{a} \cdot \hat{n} dS
                                                          &= -\iiint_V \nabla \cdot (p \vec{a}) dV
                \end{align*}
                where we used the Divergence Theorem.
                The divergence and be rewritten as:
                \begin{align*}
                    \nabla \cdot (p \vec{a}) = \vec{a} \cdot \nabla p
                \end{align*}
                since $\nabla \cdot \vec{a} = 0$ as $\vec{a}$ is a constant and does not depend on space.
                Hence, removing the scalar product with $\vec{a}$:
                \begin{align*}
                    \vec{F} = - \oiint_S p \hat{n} dS = - \iiint_V \nabla p dV
                \end{align*}
                Provided $\nabla p$ is continuous, it will be constant across an infinitesimal fluid volume $dV$.
                The net force on this fluid element due to the surround fluid is therefore the differential form:
                \begin{align*}
                    - \nabla p dV
                \end{align*}
            \end{subequations}
        \subsection{Euler's Equation}\label{subsec:euler's-equation}
            We consider the external forces per unit volume $\vec{F}_{ext}$.
            If the only external force is gravity per unit volume, $\vec{F}_{ext} = \rho \vec{g}$ where $\vec{g}$
            is the gravitational acceleration vector field.
            The total force exerted on a fluid element of volume $dV$:
            \begin{subequations}
                \begin{align*}
                    (-\nabla p + \vec{F}_{ext}) dV
                \end{align*}
                Applying Newton's second Law $\vec{F} = m\vec{a}$ where $m$ is the mass and $\vec{a}$ the acceleration.
                The total force must ths be equal to the product of the fluid mass and its acceleration:
                \begin{align*}
                    \rho dV = \frac{d \vec{u}}{d t}
                \end{align*}
                Thus,
                \begin{align*}
                    \rho \frac{d \vec{u}}{dt} = -\nabla p + \vec{F}_{ext}
                \end{align*}
                or
                \begin{align*}
                    \rho(\frac{\partial \vec{u}}{\partial t} + (\vec{u} \cdot \nabla)\vec{u}) = -\nabla p + \vec{F}_{ext}
                \end{align*}
                This is called the \textbf{Euler Equation}.
                It is a vector equation with three components.
                \\
                IF we assume an incompressible fluid, the Continuity Equation becomes $\nabla \cdot \vec{u} = 0$, which allows
                us to relate one component of the velocity as a function of the other two components.\\
                Hence we have four scalar equations with four unknowns \textbf{i.e} $u_x, u_y, u_z$ and $p$.
            \end{subequations}
\end{document}