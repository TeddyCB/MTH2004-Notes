%! Author = teddy
%! Date = 12/02/2022

% Preamble
\documentclass[11pt]{article}
\title{Integration of Vectors}
% Packages
\usepackage{amsmath}
\usepackage{esint}


% Document
\begin{document}
\maketitle
    \section{Evaluation of Line Integrals}\label{sec:evaluation-of-line-integrals}
        In a line integral, we integrate a given function $f(x,y,z)$ along a curve $C$
        in space from point $a$ at location $\vec{r(a)}$ to point $b$ at location $\vec{r(b)}$.
        \\
        In order to achieve this, we describe the curve $C$ by its parametric representation in Cartesian coordinates:
        $\vec{r(t)} = (x(t), y(t), z(t))$.
        The curve $C$ is called the path of integration. $P = \vec{r(a)}$ is its start point $Q = \vec{r(b)}$ is its end point.
        The curve $C$ is oriented positively in the direction from $P$ to $Q$ and is denoted by an arrow.
        If the points $P$ and $Q$ coincide the path is \textbf{closed}.

        \subsection{Line integral of vector field}\label{subsec:line-integral-of-vector-field}
            The line integral of a vector field $\vec{F}$ over a curve $C$ with parametric representation
            $\vec{r(t)}$ is:
            \begin{equation}
                \label{eq:equation}
                \int_{C} \vec{F}(\vec{r}) \cdot d\vec{r} = \int_{a}^{b} \vec{F}(\vec{r(t)}) \cdot \frac{d\vec{r(t)}}{dt} dt
            \end{equation}
            where $d\vec{r}$ is the curve's displacement vector or line element.
            The Line Integral can also be represented as:
            \begin{equation}
                \label{eq:equation2}
                \int_{C} \vec{F}(\vec{r}) \cdot d\vec{r} = \int_{C} F_x \: dx + F_y \: dy + F_z \: dz
            \end{equation}
        \subsection{Other forms of line integrals}\label{subsec:other-forms-of-line-integrals}
            The most common type of line integral is $\int_{C} \vec{F} \cdot d\vec{r}$.
            This line integral occurs in areas of physics to calculate the work done by a constant force field
            $\vec{F}$ in moving an object along the curve C\@.
            \\If we introduce the \textbf{unit tangent vector} $\vec{T} \equiv \frac{d\vec{r}}{ds}$ for arc length
            $s$ along the curve $C$, this i also often expressed as the line integral with respect to arc
            length of the tangential component of the vector field:
            \begin{subequations}
                \begin{align}
                    \label{eq:equation3}
                    \int_{C} \vec{F} \cdot d\vec{r} &= \int_{C} \vec{F} \cdot \vec{T} ds
                \end{align}
                \flushleft Even though integrals with respect to arc length are unchanged when orientation is
                reversed, it is still true that:
                \begin{align}
                    \int_{C} \vec{F} \cdot d\vec{r} &= - \int_{C} \vec{F} \cdot d\vec{r}
                \end{align}
                \flushleft this is because the unit tangent vector $\vec{T}$ is replaced by its negative when
                $C$ is replaced by $-C$
            \end{subequations}
            \\
            However, line integrals also come in other forms where $\phi$ is a scalar field, and $ds = | d\vec{r} |$.
            All are solved in the same way via parameterization of the curve.
            \begin{enumerate}
                \item $\int_{C} \phi \: ds$ \textbf{The result is a scalar as $\phi ds$ is a scalar}
                \item $\int_{C} \phi \: d\vec{r}$ \textbf{The result is a vector because $d\vec{r}$ is a vector}
                \item $\int_{C} \vec{F} \times d\vec{r}$ \textbf{The result is a vector as $\vec{F} \times d\vec{r}$ is a vector}
            \end{enumerate}
        \subsection{Conservative vector fields}\label{subsec:conservative-vector-fields}
            In general, a line integral depends on the vector field, the start and end points, and the shape of the path
            the curve takes.
            There are special cases when the line integral does not depend on the shape of the path taken.\\
            A vector field $\vec{F}$ is said to be \textbf{conservative} if its line integral around
            any closed curve $C$ is zero, \textbf{i.e}
            \begin{subequations}
                \label{eq:equation4}
                \begin{align}
                    \oint_{C} \vec{F} \cdot d\vec{r} = 0
                \end{align}
                \flushleft An equivalent definition is that $\vec{F}$ is conservative if the line integral along a curve
                only depends on the star and end point but not on the path taken, \textbf{i.e}:
                \begin{align}
                    \int_{C_1} \vec{F} \cdot d\vec{r} = \int_{C_2} \vec{F} \cdot d\vec{r}
                \end{align}
                \flushleft where $C_1$ and $C_2$ are any two curves that share the same start and end points
            \end{subequations}
            \subsubsection{Potential function of a vector field}
                If a vector field $\vec{F}$ is related to a scalar field $\phi$ by $\vec{F} = \nabla \phi$
                with $\nabla \phi$ existing everywhere within some region $D$, then $\vec{F}$ is conservative
                within $D$.\\
                Conversely, if $\vec{F}$ is conservative, then $\vec{F}$ can be written as the gradient of a scalar field,
                $\vec{F} = \nabla \phi$
                \paragraph{Notes}
                \begin{enumerate}
                    \item The potential $\phi$ is not unique since an arbitrary constant can be added to $\phi$ without it affecting $\vec{F} = \nabla \phi$.
                    \item The curl of a conservative field is zero: $\nabla \times \vec{F} = \nabla \times \nabla \phi = \vec{0}$.
                    Thus, a vector field with zero curl mut be conservative.
                    \item The line integral of a conservative vector field depends only on the start and end points of the curve,
                    and not on the path taken
                \end{enumerate}
        \section{Surface integrals}\label{sec:surface-integrals}
            Similarly to integrating fields over a curve yielding a line integral, we can also integrate fields over surfaces.
            To achieve this, we must first see how to represent a surface.
            \subsection{Representation of surfaces}\label{subsec:representation-of-surfaces}
                Surfaces are two-dimensional objects and require two parameters to be defined.
                We choose the name those two parameters $u$ and $v$, and define the surface as the structure traced out by the position vector:
                \begin{equation}
                    \label{eq:equation5}
                    \vec{r}(u,v) = (f(u,v), g(u,v), h(u,v))
                \end{equation}
                Where $u$ and $v$ vary in some region $R$ of the $uv$-plane.
            \subsection{Tangent plane and surface normal}\label{subsec:tangent-plane-and-surface-normal}
                If in the surface $\vec{r}(u,v)$ we fi one parameter, say $u = u_0$, and vary the other,
                then we obtain the equation of a curve on the surface.
                \textbf{e.g}
                \begin{subequations}
                    \flushleft On a spherical shell of radius $a$, we can fix $\theta = \theta_0 = \pi/4$, to obtain:
                    \begin{align}
                        \vec{r}(\theta_0, \varphi) &= (a\sin(\theta_0) \cos(\varphi), a\sin(\theta_0) \sin(\varphi), a\cos(\theta_0))
                                                  \\ &= \frac{a}{\sqrt {2}}(\cos(\varphi),\sin(\varphi),1)
                    \end{align}
                    \flushleft This is a circle of radius $a/\sqrt {2}$ on the plane $z = a/\sqrt {2}$
                \end{subequations}
                \\\\More generally, for $\vec{r}(u,v)$ we have two curves at a point $P = (u_0, v_0)$
                and we can define the tangent vectors to those curves:
                \begin{subequations}
                    \begin{align}
                        \frac{\partial \vec{r}}{\partial u} &=
                        \left( \frac{\partial f}{\partial u},
                        \frac{\partial g}{\partial u},
                        \frac{\partial h}{\partial u}\right)\\
                        \frac{\partial \vec{r}}{\partial v} &=
                        \left( \frac{\partial f}{\partial v},
                        \frac{\partial g}{\partial v},
                        \frac{\partial h}{\partial v}\right)
                    \end{align}
                    \flushleft The tangent vectors define the tangent plane to the surface at $P$.
                    The cross product of $\partial \vec{r} / \partial u$ and  $\partial \vec{r} / \partial v$
                    gives the normal vector $\vec{n}$ of the tangent plane at $P$ \textbf{i.e}
                    \begin{align}
                        \vec{n} = \frac{\partial \vec{r}}{\partial u} \times \frac{\partial \vec{r}}{\partial v}, \; \; ( \neq 0)
                    \end{align}
                    \flushleft By convention we define the \textbf{surface normal} to be the unit vector
                    locally perpendicular to the surface and pointing outwards for closed surfaces, \textbf{i.e}
                    \begin{align}
                        \hat{n} = \frac{\frac{\partial \vec{r}}{\partial u} \times \frac{\partial \vec{r}}{\partial v} }{| \frac{\partial \vec{r}}{\partial u} \times \frac{\partial \vec{r}}{\partial v} |}
                    \end{align}
                \end{subequations}
            \subsection{Surface Area}\label{subsec:surface-area}
                We already know from geometry that the area of the parallelogram spanned by two vectors
                $\vec{a}$ and $\vec{b}$ with an angle $\theta$ between them is $| \vec{a} \times \vec{b} = ab | \sin(\theta)|$.\\
                We apply this parameterised surface to define the area of the parallelogram spanned by the
                two surface tangent vectors.
                \\
                This allows us to find the area of the parameterised surface:
                \begin{equation}
                    \label{eq:equation6}
                    \text{Area} = \iint_{S} dA = \iint_{R} \left| \frac{\partial \vec{r}}{\partial u} \times \frac{\partial \vec{r}}{\partial v} \right| du \; dv
                \end{equation}
                where $R$ is the region of the $uv$-surface.
            \subsection{Explicitly defined surface by graph}\label{subsec:explicitly-defined-surface-by-graph}
                Sometimes a surface is defined explicitly tby the graph of a function.
                In this case, we parameterise the surface using $u = x$ and $v = y$:
                \begin{subequations}
                    \begin{align}
                        \vec{r}(u,v) = \vec{r}(x,y) = (x, y, f(x,y))
                    \end{align}
                    \flushleft and the tangent vectors:
                    \begin{align}
                        \frac{\partial \vec{r}}{\partial x} &= \left( 1, 0, \frac{\partial f}{\partial x} \right)\\
                        \frac{\partial \vec{r}}{\partial y} &= \left( 0, 1, \frac{\partial f}{\partial y} \right)
                    \end{align}
                    \flushleft Hence,
                    \begin{align}
                        \vec{n} &= \frac{\partial \vec{r}}{\partial x} \times \frac{\partial \vec{r}}{\partial y}
                                = \left( -\frac{\partial f}{\partial x}, -\frac{\partial f}{\partial y}, 1  \right)\\
                        \left| \frac{\partial \vec{r}}{\partial x} \times \frac{\partial \vec{r}}{\partial y} \right| &=
                        \sqrt {1 + \left( \frac{\partial f}{\partial x} \right)^2 + \left( \frac{\partial f}{\partial y} \right)^2}
                    \end{align}
                    \flushleft So that the surface element is:
                    \begin{align}
                        dA = \sqrt {1 + \left( \frac{\partial f}{\partial x} \right)^2 + \left( \frac{\partial f}{\partial y} \right)^2} dx \; dy
                    \end{align}
                    \flushleft Then the area of the surface $z = f(x,y)$ becomes
                    \begin{align}
                        \iint_{S} dA =  \iint_R \sqrt {1 + \left( \frac{\partial f}{\partial x} \right)^2 + \left( \frac{\partial f}{\partial y} \right)^2} dx \; dy
                    \end{align}
                \end{subequations}
            \subsection{Surface integral of a scalar field}\label{subsec:surface-integral-of-a-scalar-field}
                We define the integral of a scalar field $\varphi$ over the surface $S$ as:
                \begin{equation}
                    \label{eq:equation7}
                    \iint_S \varphi dA = \iint_R \varphi(\vec{r}(u, v)) \left| \frac{\partial \vec{r}}{\partial u} \times \frac{\partial \vec{r}}{\partial v} \right| du \; dv
                \end{equation}
                Note when $\varphi = 1$ we calculate the area of the parameterised surface.
            \subsection{Surface integral of a vector field and flux}\label{subsec:surface-integral-of-a-vector-field-and-flux}
                For a surface $S$ parameterised as $\vec{r}(u, v)$ and a vector field $\vec{F}(\vec{r})$,
                we define the integral of that vector field along the sruface as:
                \begin{equation}
                    \label{eq:equation8}
                    \iint_S \vec{F} \cdot \hat{n} dA = \iint_R \vec{F} \cdot \hat{n} \left| \frac{\partial \vec{r}}{\partial u} \times \frac{\partial \vec{r}}{\partial v} \right| du \; dv
                \end{equation}
                The scalar quantity $ \iint_S \vec{F} \cdot \hat{n} dA$ is called the \textbf{flux} of the vector field
                $\vec{F}$ through the surface $S$.
                \\
                \textbf{NB:} for closed surfaces we use the notation:
                \begin{equation}
                    \label{eq:equation9}
                    \oiint \vec{F} \cdot d\vec{S}
                \end{equation}
                \subsubsection{Physical interpretation}
                    Suppose a fluid flows with velocity $\vec{u}$ through a pipe with cross-sectional area A\@.
                    The \textbf{flux} through the pipe is the total volume of fluid passing through the pipe
                    per unit time, or the flux across the surface $S$ with are $A$.
                    \begin{enumerate}
                        \item Suppose the pipe's cross-sectional area is constant and equal to A, and that
                        \textbf{the velocity is uniform and directed parallel} to the walls of the pipe with
                        speed $u_0 = | \vec{u} |$.
                        Then the fluid moves along the pipe as if it were a solid block.
                        In a time interval $\Delta t$ the fluid passes a distance $u_0 \Delta t$ and so a block of fluid
                        of volume $u_0 \Delta t A$ emerges from the end of the pipe.
                        The flow rate or flux $Q$ of fluid through the pipe is therefore this volume per unit time, \textbf{i.e}
                        $Q = u_0 A$.
                        \item Suppose \textbf{the speed} of the flow is still parallel to the walls but
                        \textbf{depends on the position within the pipe} \textbf{i.e} $|\vec{u}| = u_0 (x,y)$.
                        Assume the cross-section of the pipe is a square $ 0 \leq x \leq 1$, $ 0 \leq y \leq 1$.
                        A small surface element with area $dA$ is a small rectangle with sides of lengths $dx$ and $dy$
                        , located a point $(x,y)$ on $S$, \textbf{i.e} $dA = dx \; dy$\\
                        The flux $dQ$ of fluid across the surface element $dA$ is: $dQ = u_0(x, y) dA = u_0(x, y) \; dx \: dy$.
                        The total flux follows from adding up all small contributions $dQ$:
                        \begin{align}
                            Q = \iint_S u_0 (x, y) \;dx \: dy = \int_{0}^{1} \int_{0}^{1} u_0 (x, y) \; dx \: dy
                        \end{align}
                        \item Consider the \textbf{general case} where both \textbf{velocity field} $\vec{u}$
                        \textbf{and surface} $S$ \textbf{are arbitrary} \textbf{i.e} $\vec{u}$ may not be parallel to
                        the walls and $S$ may be a non-planar surface.
                        Again, we consider the flux of $\vec{u}$ across the surface element $dA$.
                        However, $\vec{u}$ is not necessarily perpendicular to the surface $dA$.
                        Therefore, we need to extract the component of $\vec{u}$ perpendicular to the surface $dA$.
                        We make use of the unit normal vector $\hat{n}$ that is perpendicular to the surface $dA$.
                        Hence, the required component is $\vec{u} \cdot \hat{n}$.
                        The flux across surface element $dA$ is $dQ = \vec{u} \cdot \hat{n} \: dA$.
                        The total flux Q is:
                        \begin{align}
                            Q = \iint_S \vec{u}(x,y) \cdot \hat{n} \: dA
                        \end{align}
                        Since $\hat{n}$ points outwards from the pipe, a positive (negative) value of $Q$ implied fluid flowing out
                        (into) the pipe.
                    \end{enumerate}
        \section{Volume Integrals}\label{sec:volume-integrals}
            Consider an object of volume $V$ and density $\rho$.
            If $\rho$ is uniform, then the mass of the object is $M = \rho V$.
            But hat if the density is a function of position, some $\rho(\vec{r})$?
            \\
            We divide the volume into $N$ small pieces with volume $\delta V_i$, $i = 1,2,\dots,N$, called the \textbf{volume elements}.
            Then the mass of the volume element $\delta V_i$ at position $\vec{r_i}$ is simply $\delta M_i = \rho(\vec{r_i}) \delta V_i$.
            The total mass is the sum over all positions:
            \begin{subequations}
                \begin{align}
                    M = \sum_{i = 1}^{N} \delta M_i = \sum_{i=1}^{N} \rho(\vec{r}_i)\delta V_i
                \end{align}
                \flushleft In the limit of infinitesimally small volume elements and $N \rightarrow \infty$,
                we obtain a volume integral:
                \begin{align}
                    \iiint_{V} \rho dV = \lim_{N \rightarrow \infty} \sum_{i=1}^{N} \rho(\vec{r}_i)\delta V_i
                \end{align}
                \flushleft If we set $\rho = 1$, we actually calculate the total volume itself
            \end{subequations}
            \paragraph{Definition}\\
            Consider a closed surface in space enclosing a volume $V$, then the following \textbf{volume integrals, space integrals}
            or \textbf{triple integrals}:
            \begin{equation}
                \label{eq:equation10}
                \iiint_{V} \vec{A} \; dV \text{    and     } \iiint_{V} \phi \; dV
            \end{equation}
            where $\vec{A}$ is a vector field, $\phi$ is a scalar field, and $V$ is a 3D volume.
            \subsection{Volume elements in different coordinate systems}\label{subsec:volume-elements-in-different-coordinate-systems}
                \begin{enumerate}
                    \item \textbf{Cartesian coordinates} $(x,y,z)$: $dV = dx dy dz$
                    \item \textbf{Cylindrical coordinates} $(R,\varphi,z)$:  $dV = R dR d\varphi dz$
                    \item \textbf{Spherical coordinates} $(r, \theta, \phi)$: $dV = r^2 \sin(\theta) dr d\theta d\varphi$
                \end{enumerate}
            \subsection{The Jacobian}\label{subsec:the-jacobian}
                We already know from geometry that the volume of the parallelepiped spanned by three vectors $\vec{a}$,
                $\vec{b}$ and $\vec{c}$ is $|(\vec{a} \times \vec{b}) \cdot \vec{c}$.
                Applying this to the parameterised volume to define the volume of the parallelepiped spanned by three surface tangent
                vectors we get the \textbf{Jacobian}:
                \begin{subequations}
                    \begin{align}
                        J = \frac{\partial(x,y,z)}{\partial(u,v,w)} = \left \left( \frac{\partial \vec{r}}{\partial u} \times \frac{\partial \vec{r}}{\partial v} \right)  \cdot \frac{\partial \vec{r}}{\partial w}\right|
                        = \begin{vmatrix}
                              \frac{\partial x}{\partial u} & \frac{\partial y}{\partial u} & \frac{\partial z}{\partial u} \\
                              \frac{\partial x}{\partial v} & \frac{\partial y}{\partial v} & \frac{\partial z}{\partial v} \\
                              \frac{\partial x}{\partial w} & \frac{\partial y}{\partial w} & \frac{\partial z}{\partial w}
                        \end{vmatrix}
                    \end{align}
                    \flushleft This allows us to write a volume, or triple, integral originally written in terms of Cartesian coordinates
                    \begin{align*}
                        \iiint_{V} f(x,y,z) \; dx dy dz
                    \end{align*}
                    \flushleft instead in terms of another coordinate system $(u,v,w)$:
                    \begin{align*}
                        \iiint_{V} f(u,v,w) J \; du dv dw
                    \end{align*}
                    \flushleft where $f$, $J$ and the integration limits must be in terms of the new variables $u$, $v$ and $w$
                \end{subequations}
        \section{Integral Theorems}\label{sec:integral-theorems}
            There are two important theorems that link line, surface and volume integrals with the definitions of the divergence and curl 
            operators.
            These theorems have significance in deriving mathematical equations representing physical laws.
            \subsection{Divergence Theorem}\label{subsec:divergence-theorem}
                The transformation between \textbf{triple} and \textbf{surface} integrals, which involves the \textbf{divergence} of a vector function $\vec{F}$.
                \paragraph{Theorem} Consider a continuously differentiable vector field $\vec{F}$ defined on a volume $V$ that is enclosed by a surface $S$.
                Then:
                \begin{equation}
                    \label{eq:equation11}
                    \iiint_{V} \nabla \cdot \vec{F} \; dV = \oiint_{S} \vec{F} \cdot d\vec{S}
                \end{equation}
                where $d\vec{S} = \hat{n} dS$.\\
            \textbf{This theorem says that the total expansion of a vector field over a volume is equal to the flux
            of that vector field through the bounding surface.}
            \subsection{Stokes' Theorem}\label{subsec:stokes'-theorem}
                This describes the transformation between \textbf{surface} and \textbf{line} integrals, which involves
                the \textbf{curl} of a vector function $\vec{F}$
                \paragraph{Theorem} Consider a continuously differentiable vector field $\vec{F}$ defined
                within an open surface $S$.
                Then,
                \begin{equation}
                    \label{eq:equation12}
                    \iint_{S} \nabla \times \vec{F} \cdot d\vec{S} = \oint_{C} \vec{F} \cdot d\vec{r}
                \end{equation}
                where $d\vec{S} = \hat{n} dS$ and the direction of the line integral around $C$ and the unit
                normal $\hat{n}$ are oriented in a right-hand sense
                \textbf{This theorem says that the total flux of curl of a vector field through a surface is equal
                to the closed line integral of that vector field around the surface boundary}
\end{document}