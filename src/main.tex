%! Author = teddy
%! Date = 17/01/2022

% Preamble
\documentclass[11pt]{article}
\title{MTH2004 - Vector Calculus and Applications}
\author{Teddy Cameron- Burke}
% Packages
\usepackage{amsmath}
\usepackage{wasysym}
\usepackage{amssymb}
\usepackage{xcolor}

% Document
\begin{document}
\maketitle
\section{Introduction and Preliminaries}\label{sec:introduction-and-preliminaries}
\subsection{Vectors}\label{subsec:vectors}
A vector describes a quantity that has a magnitude and direction in three-dimensional space.
A special type of vector is a \textbf{position vector} that always starts at the origin of the coordinate system and points to the position in question.
A \textbf{unit vector} is a position vector of length one that defines the directions of the coordinate system.\\
\textbf{Graphically}, a vector is represented by an arrow that starts at a position, e.g.\ centre of mass, origin, and point in the correct direction.\\
Symbolically, a vector in this module is depicted as $\vec{x}$ and a unit vector as $\hat{x}$.\\
\textbf{Mathematically}, a vector is defined by its decomposition in the coordinate system.
We will always assume orthogonal coordinate systems.
In Cartesian coordinates $x,y,z$ then:
   \begin{equation}
       \vec{v} = v_{x}\hat{x} + v_y\hat{y} + v_z\hat{z}\label{eq:equation} 
   \end{equation}
where $v_x, v_y, v_z$ are the $x, y, z$ components of the vector.\\
\paragraph{Multiplication with a scalar}
Multiplying a vector with a scalar is multiplying each component.
The result is still a vector:
    \begin{equation}
        \label{eq:equation2}
        \alpha \vec{v} = \alpha v_x \hat{x} + \alpha v_y \hat{y} + \alpha v_z \hat{z}
    \end{equation}
\paragraph{Adding or subtracting vectors}
Two vectors can be summed or subtracted from each other by adding or subtracting each component.
This also results in a vector:
    \begin{equation}
        \label{eq:equation3}
        \vec{a} \pm \vec{b} = (a_x \pm b_x)\hat{x} + (a_y \pm b_y)\hat{y} + (a_z \pm b_z)\hat{z}
    \end{equation}
\paragraph{Magnitude of a vector}
For a position vector, it's magnitude equals the distance from the origin to the position.
The result is a scalar:
\begin{equation}
    \label{eq:equation4}
    |\vec{v}| = \sqrt {v^2_x + v^2_y + v^2_z}
\end{equation}
\paragraph{Unit vector in direction of vector}
We can define a vector of unit length in the direction of a vector $\vec{v}$ as:
    \begin{equation}
        \label{eq:equation5}
        \hat{v} = \frac{\vec{v}}{|\vec{v}|}
    \end{equation}
\textbf{NB}: The unit vector does not have to start at the origin!
\paragraph{Scalar Product}
The scalar or dot product between two vectors is defined as the product between the magnitude of one vector and the projection of the other vector onto it:\\
\begin{equation}
    \label{eq:equation6}
    \vec{a} \cdot \vec{b} = |\vec{a}|\:|\vec{b}|\cos \alpha
\end{equation}
Note that $\vec{a} \cdot \vec{b} = \vec{b} \cdot \vec{a}$.
Further, perpendicular vectors result in $\vec{a} \cdot \vec{b} = 0$.
\paragraph{Cross (or Vector) Product}
The Cross Product between two vectors produces a third vector that is perpendicular to both according to the right-hand rule.\\
\begin{equation}
    \label{eq:equation7}
    \vec{a} \times \vec{b} =
    \begin{vmatrix}
        \hat{x} & \hat{y} & \hat{x} \\
        a_x & a_y & a_z \\
        b_x & b_y & b_z
    \end{vmatrix}
    = |\vec{a}||\vec{b}|\sin \alpha
\end{equation}
Note that $\vec{a} \times \vec{b} = - \vec{b} \times \vec{a}$ and also that $| \vec{a} \times \vec{b} |$ is the
`surface area` of the parallelogram made by $\vec{a}$ and $\vec{b}$.\\
Further note that \textbf{if two vectors are parallel then their cross product is zero}.
\paragraph{Scalar Triple Product}
The volume of the parallelepiped formed by $\vec{c}$, $\vec{a}$ and $\vec{b}$ is the magnitude of the scalar:\\
\begin{equation}
    \label{eq:equation8}
    \vec{c} \cdot \vec{a} \times \vec{b} =
    \begin{vmatrix}
        c_x & c_y & c_z \\
        a_x & a_y & a_z \\
        b_x & b_y & b_z
    \end{vmatrix}
    = |\: \vec{c}\:|  |\vec{a} \times \vec{b}| \cos \theta
\end{equation}
where $\theta$ is the angle between $\vec{c}$ and $\vec{a} \times \vec{b}$.\\
\\Note that $\vec{c} \cdot \vec{a} \times \vec{b} = \vec{a} \cdot \vec{b} \times \vec{c} = \vec{b} \cdot \vec{c} \times \vec{a}$.
Further note that \textbf{three vectors are linearly independent if and only if their scalar triple product is non-zero}.
\paragraph{Vector Triple Product}
The vector $\vec{a} \times (\vec{b} \times \vec{c})$ is perpendicular to both $\vec{a}$ and $\vec{b} \times \vec{c}$.
Since the plane defined by $\vec{b}$ and $vec{c}$ is perpendicular to $\vec{b} \times \vec{c}$, the triple product lies in this
plane.
\subsection{Differentiation Methods}\label{subsec:differentiation-methods}
\paragraph{Product Rule}
For two functions $f{t}$ and $g{t}$:
    \begin{equation}
        \label{eq:equation9}
        \frac{d}{dt}(fg) = f\frac{dg}{dt} + g \frac{df}{dt}
    \end{equation}
\paragraph{Chain Rule}
For a function $f(g(t))$:
    \begin{equation}
        \label{eq:equation10}
        \frac{d}{dt}(f(g(t))) = \frac{df}{dg}\frac{dg}{dt}
    \end{equation}
\paragraph{Integration by Parts}
For two functions $u(t)$ and $v'(t)$
    \begin{equation}
        \label{eq:equation11}
        \int_{a}^{b} uv' \: dt = [uv]_{a}^{b} - \int_{a}^{b} u'v \: dt
    \end{equation}
\subsection{Suffix Notation}\label{subsec:suffix-notation}
Many vector expressions can be simplified and more easily derived if we introduce the suffix notation.
    \paragraph{Summation Convention}
    A suffix, that appears twice and no more within a term implies that the term is to be summed from $i = 1$ to $3$.
    This repeated suffix is also referred to as a dummy suffix.\\
    e.g.
    \begin{subequations}
        \begin{equation}
            \label{eq:equation12}
            \vec{a} \cdot \vec{b} = \sum_{i = 1}^{3} a_{i}b_i = a_{i}b_i
        \end{equation}
        \begin{equation}
            \label{eq:equation13}
            (\vec{a} \cdot \vec{b})(\vec{c} \cdot \vec{d}) = \sum_{i = 1}^{3}\sum_{j = 1}^{3} a_i b_i c_j d_j
            = a_i b_i c_j d_j = a_i c_j b_i d_j
        \end{equation}
    \end{subequations}
    \paragraph{Kronecker Delta}
    The Kronecker delta is defined as:
    \begin{equation}
        \label{eq:equation14}
        \delta_{ij} =
        \begin{cases}
            1 & \text{if $i = j$} \\
            0 & \text{if $i \neq j$}
        \end{cases}
    \end{equation}
    The Kronecker delta also can represents the 3x3 unit matrix:
    \begin{equation}
        \label{eq:equation15}
        \delta_{ij} =
        \begin{bmatrix}
            1 & 0 & 0 \\
            0 & 1 & 0 \\
            0 & 0 & 1
        \end{bmatrix}
    \end{equation}
    and this is symmetric, i.e. $\delta_{ij} = \delta_{ji}$

    \paragraph{Kronecker delta and summation notation}
    Consider $\delta_{ij} a_j = \sum_{j = 1}^{3} \delta_{ij} a_j$.
    For $i = 1$ only $\delta_{i1} \neq 0$, hence $\delta_{1j} a_j = a_1$.
    Also, $\delta_{2j} a_j = a_2$ and $\delta_{3j} a_j = a_3$.
    Thus:\\
    \begin{equation}
        \label{eq:equation16}
        \delta_{ij} a_j = a_i
    \end{equation}
    \paragraph{Alternating Tensor}
    The alternating tensor (or permutation tensor) $\epsilon_{ijk}$ is defined as:
    \begin{equation}
        \label{eq:equation17}
        \epsilon_{ijk} =
        \begin{cases}
            1 & \text{if $(i,j,k) = (1,2,3),(2,3,1)$ or $(3,1,2)$}\\
            -1 & \text{if $(i,j,k) = (1,3,2), (2,1,3)$ or $(3,2,1)$}\\
            0 & \text{if any of $i,j$ or $k$ are equal}
        \end{cases}
    \end{equation}
    The alternating tensor also represents a 3x3x3 object with 27 elements of which ony 6 are non zero:
    \begin{equation}
        \label{eq:equation18}
        \epsilon_{123} = \epsilon_{231} = \epsilon_{312} = 1, \:
        \epsilon_{132} = \epsilon_{213} = \epsilon_{321} = -1
    \end{equation}
    Importantly, $\epsilon_{ijk}$ remains unchanged if the suffixes are reordered by shifting to the right and putting the last
    suffix first, or by shifting to the left and putting the first suffix last:
    \begin{equation}
        \label{eq:equation19}
        \epsilon_{ijk} = \epsilon_{kij} = \epsilon_{jki}
    \end{equation}
    Further, the sign of the tensor changes if two suffixes are interchanged:
    \begin{equation}
        \label{eq:equation20}
        \epsilon_{ijk} = -\epsilon_{ikj}
    \end{equation}
    Also, the alternating tensor is useful for expressing cross products:
    \begin{equation}
        \label{eq:equation21}
        (\vec{a} \times \vec{b})_i = \sum_{j = 1}^{3} \sum_{k = 1}^{3} \epsilon_{ijk} a_j b_k = \epsilon_{ijk} a_j b_k
    \end{equation}
    \textbf{Proof:}\\
    \begin{subequations}
        To verify this, consider the case $i = 1$:
        \begin{equation}
            \label{eq:equation22}
            (\vec{a} \times \vec{b})_1 = \epsilon_{1jk} a_j b_k = \sum_{j = 1}^{3} \sum_{k = 1}^{3} \epsilon_{1jk} a_j b_k
        \end{equation}
        \\The only non-zero contributions to $\epsilon_{1jk}$ are for values $j = 2, \; k = 3$ and $j = 3, \; k = 2$.
        Hence:\\
        \begin{equation}
            \label{eq:equation23}
            (\vec{a} \times \vec{b} )_1 = \epsilon_{123} a_2 b_3 + \epsilon_{132} a_3 b_2 = a_{2} b_3 - a_3 b_2
         \end{equation}
        Similarly for $i = 2,3$
    \end{subequations}
    A useful equivalence is the scalar triple product in suffix notation, which can be written as:
    \begin{equation}
        \label{eq:equation24}
        \vec{a} \cdot(\vec{b} \times \vec{c}) = a_i ( \vec{b} \times \vec{c})_i = \epsilon_{ijk} a_i b_j c_k
    \end{equation}
    \subsection{Relationship between $\delta_{ij}$ and $\epsilon_{ijk}$}\label{subsec:relationship-between-$delta_{ij}$-and-$epsilon_{ijk}$}
    Consider the following relationship:\\
    \begin{equation}
        \label{eq:equation25}
        \epsilon_{ijk} \epsilon_{klm} = \delta_{il} \delta_{jm} - \delta_{im} \delta_{jl}
    \end{equation}
    Note that $k$ is a dummy suffix as it appears twice, and there are four free suffixes: $i, \: j, \: k, \: l$ and $m$.
    Therefore, this expression represents $3^4 = 81$ different equations.\\
    \textbf{Proof:}\\
    \begin{subequations}
        We only need to consider one case, e.g $i = 1$, since the three coordinate axes are equivalent.
        Consider the possible values for $j$.\\
        \textbullet \;If $j = 1$, $\epsilon_{ijk} = \epsilon_{11k} = 0$, and so the L.H.S is zero the R.H.S is
        $\delta_{1l} \delta_{1m} - \delta_{1l}$ which is also zero.\\
        \textbullet \;If $j = 2$ $\epsilon_{ijk} = \epsilon_{12k} = 0$ unless $ k = 3$ then $\epsilon_{ijk} = \epsilon_{123} = 1$.
        Therefore only the $k = 3$ terms contribute to the sum.
        When $k = 3$, the term $\epsilon_{klm} = 0$ unless $l$ and $m$ are 1 and 2.
        Therefore, the L.H.S takes the value +1 if $l = 1$ and $ m = 2$, -1 if $l = 2$ and $m = 1$, and zero otherwise.
        The R.H.S is $\delta_{1l} \delta_{2m} - \delta_{1m} \delta_{2l}$ which also has the same equality for the given values of $m$ and $l$.\\
        \textbullet \; If $j = 3$, similar arguments as for $j = 2$ apply.
    \end{subequations}
    \newline
    \textbf{NB:} This relation will be useful whn considering terms involving two vector cross products.
\end{document}