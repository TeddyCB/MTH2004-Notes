%! Author = teddy
%! Date = 07/02/2022

% Preamble
\documentclass[11pt]{article}
\title{Differentiation of Vectors}
% Packages
\usepackage{amsmath}
\usepackage{graphicx}
\usepackage{pgfplots}
\usepackage{tikz}
\usetikzlibrary{calc}

\pgfplotsset{compat= newest}

% Document
\begin{document}
\maketitle

    \section{Differentiation Rules for Vector Fields}\label{sec:differentiation-rules-for-vector-fields}
    \subsection{Cartesian coordinates}\label{subsec:cartesian-coordinates}
        For a vector function or field $\vec{A} = A_x \hat{x} +  A_y \hat{y}+  A_z \hat{z} $.
        If the vector function depends on one variable, then it is differentiated component wise:
        \begin{equation}
            \label{eq:equation}
            \frac{d\vec{A}}{dt} = \frac{dA_x}{dt}\hat{x} + \frac{dA_y}{dt}\hat{y} + \frac{dA_z}{dt}\hat{z}
        \end{equation}
        Vector differentiation is also linear for multiplication and addition.
        Importantly for differentiation on field operators $\cdot$ and $\times$ is:
        \begin{subequations}
            \begin{center}
                \begin{align}
                    \frac{d}{dt} (\vec{A} \cdot \vec{B}) = \frac{d\vec{A}}{dt} \cdot \vec{B} + \vec{A} \cdot \frac{d\vec{B}}{dt}\\
                    \frac{d}{dt} (\vec{A} \times \vec{B}) = \frac{d\vec{A}}{dt} \times \vec{B} + \vec{A} \times \frac{d\vec{B}}{dt}
                \end{align}
            \end{center}
        \end{subequations}
        \subsubsection{Curvilinear coordinates}
            Consider cylindrical coordinates $(R,\varphi)$.
            The polar coordinate base vectors $\hat{R}$ and $\hat{\varphi}$ are constant in magnitude, but
            they change their directions from point to point, depending on the polar angle $\varphi$.\\
            \paragraph{Example} Consider the vector components: $\hat{R} = \cos (\varphi) \hat{x} + \sin (\varphi) \hat{y}$,
            $\hat{\varphi} = -\sin(\varphi) \hat{x} + \cos(\varphi) \hat{y}$ with $\varphi(t)$.
            Then:
            \begin{subequations}
                \begin{center}
                    \begin{align}
                        \frac{d\hat{R}}{dt} &= \frac{d}{dt}(\cos(\varphi)\hat{x}) + \frac{d}{dt}(\sin(\varphi)\hat{y})\\
                                            &= \frac{d\cos(\varphi)}{dt}\hat{x} + \cos(\varphi)\frac{d\hat{x}}{dt} + \frac{d\sin(\varphi)}{dt}\hat{y} + \sin(\varphi)\frac{d\hat{y}}{dt}\\
                                            &= -\sin(\varphi) \frac{d\varphi}{dt}\hat{x} + \cos(\varphi) \frac{d\varphi}{dt}\hat{y}\\
                                            &\Rightarrow \frac{d\hat{R}}{dt} = \frac{d\varphi}{dt}\hat{\varphi}
                    \end{align}
                \end{center}
            \end{subequations}
            Similarly, $\frac{d\varphi}{dt} = -\frac{d\varphi}{dt}\hat{R}$.
    \section{Differential operators for vector fields}\label{sec:differential-operators-for-vector-fields}
        \subsection{Divergence}\label{subsec:divergence}
        The divergence of a vector field $\vec{F} = F_x \hat{x} + F_y \hat{y} + F_z \hat{z}$ is defined as:
        \begin{equation}
            \label{eq:equation2}
            \text{div } \vec{F} = \nabla \cdot \vec{F} = \frac{\partial F_x}{\partial x} + \frac{\partial F_y}{\partial y} + \frac{\partial F_z}{\partial z}
        \end{equation}
        We can derive this operation from taking the scalar product of the gradient operator acting on the vector field
        $\vec{F}$.\\
        \textbf{NB:} The divergence of a vector field yields a scalar!
        \paragraph{Physical interpretation} The divergence of a vector field measures the rate of expansion or diverging of the
        vector field.
        For example, the velocity of a gas. $\vec{u}$, with typical flow speed constant $u_0$ and length constant $L$.
        Then:
        \begin{enumerate}
            \item $\vec{u} = \frac{u_0\;}{L} \; x \hat{x}$ $\Rightarrow$ Gas expands away from y-axis with $ \nabla \cdot \vec{u} = 1 \times \frac{u_0}{L}$
            \item $\vec{u} = -\frac{u_0\;}{L} \; x \hat{x}$ $\Rightarrow$ Gas contracts towards from y-axis with $ \nabla \cdot \vec{u} = -1 \times \frac{u_0}{L}$
            \item $\vec{u} = \frac{u_0\;}{L} \; \vec{r}$ $\Rightarrow$ Gas expands radially out in 3d with $ \nabla \cdot \vec{u} = 3 \times \frac{u_0}{L}$
            \item $\vec{u} = \frac{u_0}{L}({y\hat{x} + x \hat{y}})$ $\Rightarrow$ Gas has $\nabla \cdot \vec{u} = 0$ everywhere
        \end{enumerate}
        A vector field that has zero divergence everywhere is said to be solenoidal or divergence-free (in fluid dynamics, incompressible).
        \subsection{The curl operator}\label{subsec:the-curl-operator}
        We define the curl of a vector field $\vec{F} = F_x \hat{x} + F_y \hat{y} + F_z \hat{z}$ as:
        \begin{equation}
            \label{eq:equation3}
            \text{curl } \vec{F} = \nabla \times \vec{F} = \left(\frac{\partial F_z}{\partial y} - \frac{\partial F_y}{\partial z}\right)\hat{x} + \left(\frac{\partial F_x}{\partial z} - \frac{\partial F_z}{\partial x}\right)\hat{y} + \left(\frac{\partial F_y}{\partial x} - \frac{\partial F_x}{\partial y}\right)\hat{z}
        \end{equation}
        Derived by taking the vector (cross) product of the gradient with the vector $\vec{F}$.\\
        \textbf{NB: } The curl of a vector field yields another vector field! \\
        Further, a vector field for which $\nabla \times \vec{F} = \vec{0}$ everywhere is irrotational.\\

        \paragraph{Physical interpretation} The curl of a vector field is related to the rotation or twisting of a vector field.
        Consider again the velocity field of a gas $\vec{u}$ as before.
        Then:
        \begin{enumerate}
            \item $\vec{u} = \frac{u_0\;}{L} \; x \hat{x}$ $\Rightarrow$ is irrotational because $ \nabla \times \vec{u} = \vec{0}$
            \item $\vec{u} = \frac{u_0\;}{L} \; \vec{r}$ $\Rightarrow$ is irrotational because $ \nabla \times \vec{u} = \vec{0}$
            \item $\vec{u} = \frac{u_0}{L}({y\hat{x} + x \hat{y}})$ is irrotational because $ \nabla \times \vec{u} = \vec{0}$
            \item $\vec{u} = \frac{u_0}{L}({-y\hat{x} + x \hat{y}})$ rotates around the z-axis as $\nabla \times \vec{u} = \frac{2u_0}{L}\hat{z}$
       \end{enumerate}
        \subsection{The $\vec{F} \cdot \nabla$ operator}\label{subsec:the-$vec{f}-cdot-nabla$-operator}
            The operator $\vec{F} \cdot \nabla$ is defined as:
            \begin{equation}
                \label{eq:equation4}
                \vec{F} \cdot \nabla = F_x \frac{\partial}{\partial x} + F_y \frac{\partial}{\partial y} + F_z \frac{\partial}{\partial z}
            \end{equation}
            Importantly, the operator is not distributive and can act on a scalar or vector field
            \paragraph{Physical interpretation} Consider a velocity vector field rotating anti-clockwise, i.e. $\vec{u} = \frac{u_0}{L}(-y\hat{x} + x\hat{y})$\\
            We calculate:
            \begin{equation}
                \label{eq:equation5}
                (\vec{u} \cdot \nabla) \vec{u} = -\frac{u^2_0}{L^2}(x\hat{x} + y\hat{y})
            \end{equation}
            This points radially inwards, and is proportional to the centripetal acceleration that bodies feel when following
            a curved path.
            \subsection{The Laplacian}\label{subsec:the-laplacian}
                Recall that the gradient of a scalar $\phi(x,y,z)$ is a vector field $\nabla phi$.
                We can then take the divergence of $\nabla \phi$ to obtain:
                \begin{equation}
                    \label{eq:equation6}
                    \nabla^2 \phi = \nabla \cdot \nabla \phi = \frac{\partial^2 \phi}{\partial x^2} + \frac{\partial^2 \phi}{\partial y^2} + \frac{\partial^2 \phi}{\partial z^2}
                \end{equation}
                Note that it is also possible to take the Laplacian of a vector field:
                \begin{equation}
                    \label{eq:equation7}
                    \nabla^2 \vec{F} = \nabla^2 F_x \hat{x} + \nabla^2 F_y \hat{y} + \nabla^2 F_z \hat{z}
                \end{equation}

                \paragraph{Physical interpretation} The Laplacian gives the differnece between the average value of a function
                in the neighborhood of a point, and its value at that point.
                Thus, in the case of the temperature, the Laplacian tells whether the material surrounding each point is hotter or colder,
                on the average, than the material at that point.\\
                Consider the temperature $T$ in a space.
                Any variations in temperature leads to a thermal flux that will even out these variations.
                The rate of change of temperature over time is:
                \begin{equation}
                    \label{eq:equation8}
                    \frac{\partial T}{\partial t} = \kappa \nabla^2 T
                \end{equation}
                At a maximum temperature, the shape is convex and the slope of $T$ decreases in all directions, hence
                $\nabla^2 T < 0$.
                The rate of temperature $\frac{\partial T}{\partial t} < 0$ and the temperature will decrease.\\
                At a minimum temperature, the shape is concave and the slope of $T$ increases in all directions, hence
                $\nabla^2 T > 0$.
                The rate of temperature $\frac{\partial T}{\partial t} > 0$ and the temperature will increase
            \subsection{Differential operators in suffix notation}\label{subsec:differential-operators-in-suffix-notation}
                We label the Cartesian coordinates $(x,y,z)$ as $(x_1,x_2,x_3)$
                \begin{subequations}
                    \begin{center}
                        \begin{align}
                            \nabla = \left( \frac{\partial}{\partial x_1}, \frac{\partial}{\partial x_2}, \frac{\partial}{\partial x_3} \right)
                        \end{align}
                        \flushleft {or in suffix notation for the $i$th component:}
                        \begin{align}
                        (\nabla)_i = \frac{\partial}{\partial x_i}
                        \end{align}
                    \end{center}
                \end{subequations}
                For the Gradient:
                \begin{equation}
                    \label{eq:equation9}
                    \nabla \phi = \hat{x_i} \frac{\partial \phi}{\partial x_i}
                \end{equation}
            \subsection{Vector operator identities}\label{subsec:vector-operator-identities}
                \begin{figure}[htp]
                    \label{fig:figure}
                    \includegraphics[scale =.75]{vector_diff_rules}
                \end{figure}
            \subsection{Differential operators in orthogonal curvilinear coordinates}\label{subsec:differential-operators-in-orthogonal-curvilinear-cooridnates}
                In many physical problems the geometry is such that other coordinate systems are more convenient than Cartesian
                coordinates.
                \subsubsection{Scale Factors}
                    Consider a transformation from Cartesian coordinates $(x_1,x_2,x_3)$ to another coordinate system $(u_1,u_2,u_3)$
                    called the curvilinear coordinate system.
                    There is a one-to-one correspondence between the two coordinate systems, such that:
                    \begin{equation}
                        \label{eq:equation10}
                        x_i = x_i (u_1,u_2,u_3) \; \: \: u_i = u_i(x_1,x_2,x_3)
                    \end{equation}
                    Surfaces where $u_i$ is constant are coordinate surfaces.
                    The intersection of two coordinate surfaces define coordinate curves.\\
                    Consider a small displacement $d\vec{r} = (dx_1,dx_2,dx_3)$.
                    Since $x_i$ are a function of $u_i$, we may employ the chain rule:
                    \begin{equation}
                        d\vec{r} = \frac{\partial\vec{r}}{\partial u_1} du_1 + \frac{\partial\vec{r}}{\partial u_2} du_2 + \frac{\partial\vec{r}}{\partial u_3} du_3
                    \end{equation}
                    Note that the partial derivative $\frac{\partial \vec{r}}{\partial u_1}$ donates the variation with respect to
                    $u_1$ with respect with $u_2$ and $u_3$ kept constant.
                    Thus, $\frac{\partial \vec{r}}{\partial u_1}$ must lie in the $u_2$ ad $u_3$ coordinate surfaces and parallel with
                    the $u_1$ coordinate curve.\\
                    We define the unit vector in the direction of each coordinate curve as:
                    \begin{subequations}
                        \begin{center}
                            \begin{align}
                                \hat{e_1} = \frac{1}{h_1}\frac{\partial \vec{r}}{\partial u_1}    && \text{with } h_1 = \left| \frac{\partial \vec{r}}{\partial u_1} \right| \\
                                \hat{e_2} = \frac{1}{h_2}\frac{\partial \vec{r}}{\partial u_2}    && \text{with } h_2 = \left| \frac{\partial \vec{r}}{\partial u_2} \right| \\
                                \hat{e_1} = \frac{1}{h_3}\frac{\partial \vec{r}}{\partial u_3}    && \text{with } h_3 = \left| \frac{\partial \vec{r}}{\partial u_3} \right|
                            \end{align}
                        \end{center}
                    \end{subequations}
                    The quantities $h_i$ are called \textbf{scale factors}.\\\\
                    We will only consider orthogonal curvilinear coordinate systems (they are perpendicular to each other where they intersect).
                    Further, the basis vectors $\hat{e_i}$ are also unit vectors.
                    Thus:
                    \begin{equation}
                        \label{eq:equation11}
                        \hat{e_i} \cdot \hat{e_j} = \delta_{ij}
                    \end{equation}
                    We will further assume that the coordinate system will be right-handed, thus:
                    \begin{subequations}
                        \begin{align}
                            \hat{e_1} \times \hat{e_2}  = \hat{e_3} && \text{, } \hat{e_2} \times \hat{e_3}  = \hat{e_1} && \text{, } \hat{e_3} \times \hat{e_1}  = \hat{e_2}
                        \end{align}
                    \end{subequations}
                    From this, we can rewrite the displacement as:
                    \begin{equation}
                        \label{eq:equation12}
                        d\vec{r} = h_1 u_1 \hat{e_1} + h_2 u_2 \hat{e_2} + h_3 u_3 \hat{e_3}
                    \end{equation}
                    For the gradient of a scalar field in curvilinear coordinates:
                    \begin{equation}
                        \label{eq:equation13}
                        \nabla f = \frac{1}{h_1} \frac{\partial f}{\partial u_1} \hat{e_1}
                        + \frac{1}{h_2} \frac{\partial f}{\partial u_2} \hat{e_2}
                         + \frac{1}{h_3} \frac{\partial f}{\partial u_3} \hat{e_3}
                    \end{equation}
                    Divergence is thus:
                    \begin{equation}
                        \label{eq:equation14}
                        \nabla \cdot \vec{F} = \frac{1}{h_1 h_2 h_3}\left( \frac{\partial F_1 h_2 h_3}{\partial u_1}
                        + \frac{\partial F_2 h_3 h_1}{\partial u_2}
                        + \frac{\partial F_3 h_1 h_2}{\partial u_3}
                        \right)
                    \end{equation}
                    Curl is then:
                    \begin{equation}
                        \label{eq:equation15}
                        \nabla \times \vec{F} = \frac{1}{h_1 h_2 h_3}
                        \begin{vmatrix}
                            h_1 \hat{e_1} && h_2 \hat{e_2} && h_3 \hat{e_3} \\
                            \frac{\partial}{\partial u_1} &&  \frac{\partial}{\partial u_2} &&  \frac{\partial}{\partial u_3}\\
                            h_1 F_1 && h_2 F_2 && h_3 F_3
                        \end{vmatrix}
                    \end{equation}
                    The Laplacian can be found by using the definition of divergence with $\vec{F} = \nabla f$
                    \begin{equation}
                        \label{eq:equation16}
                        \nabla^2 f = \frac{1}{h_1 h_2 h_3} \left[ \frac{\partial}{\partial u_1} \left( \frac{h_2 h_3}{h_1}
                        \frac{\partial f}{\partial u_1}\right)
                        + \frac{\partial}{\partial u_2} \left( \frac{h_3 h_1}{h_2}
                        \frac{\partial f}{\partial u_2}\right)
                        +\frac{\partial}{\partial u_3} \left( \frac{h_1 h_2}{h_3}
                        \frac{\partial f}{\partial u_3}\right)
                        \right]
                    \end{equation}
                    For the Laplacian of the vector field, use the identity:
                    \begin{equation}
                        \label{eq:equation17}
                        \nabla^2 \vec{F} = \nabla ( \nabla \cdot \vec{F}) - \nabla \times (\nabla \times \vec{F})
                    \end{equation}
                \subsubsection{Cylindrical (polar) coordinates $(R, \varphi, z)$}
                    The Cartesian and cylindrical coordinates are related by:
                    \begin{equation}
                        \label{eq:equation18}
                         x = R \cos (\varphi), \; \; y = R \sin (\varphi), \; \; z = z
                    \end{equation}
                    The position vector $\vec{r}$ can be written as:
                    \begin{equation}
                        \label{eq:equation19}
                        \vec{r} = R \cos (\varphi) \hat{x} + R \sin (\varphi) \hat{y} + z \hat{z}
                    \end{equation}
                    The scale factors in polar coordinates are:
                    \begin{subequations}
                        \begin{align}
                            h_R = 1 \\
                            h_{\varphi} = R \\
                            h_z = 1 \\
                        \end{align}
                        \flushleft The basis vectors are:
                        \begin{align}
                            \hat{e_R} = \hat{R} \\
                            \hat{e_\varphi} = \hat{\varphi} \\
                            \hat{z} = \hat{z}
                        \end{align}
                    \end{subequations}
                    \paragraph{Differential operators:}
                    \begin{subequations}
                        \begin{align}
                            \nabla f  &= \frac{\partial f}{\partial R} \hat{e_R} + \frac{1}{R}\frac{\partial f}{\partial \varphi} \hat{e_{\varphi}}
                            + \frac{\partial f}{\partial z} \hat{e_z}\\
                             \nabla \cdot \vec{F} &= \frac{1}{R}\frac{\partial}{\partial R}(RF_R)
                            + \frac{1}{R}\frac{\partial F_{\varphi}}{\partial \varphi} + \frac{\partial F_z}{\partial z}\\
                             \nabla \times \vec{F} &= \frac{1}{R}
                            \begin{vmatrix}
                                \hat{e_R} && R \hat{e_\varphi} && \hat{e_z} \\
                                \frac{\partial}{\partial R} && \frac{\partial}{\partial \varphi} && \frac{\partial}{\partial z} \\
                                F_R && R F_{\varphi} && F_z
                            \end{vmatrix}\\
                            \nabla^2 f &= \frac{1}{R}\frac{\partial}{\partial R}\left(
                            R \frac{\partial f}{\partial R}\right) + \frac{1}{R^2}\frac{\partial^2 f}{\partial \varphi^2}
                            + \frac{\partial^2 f}{\partial z^2}
                        \end{align}
                    \end{subequations}
                \subsubsection{Spherical (polar) coordinates}
                    The Cartesian and spherical coordinates are related via:
                    \begin{equation}
                        \label{eq:equation20}
                        x = r \sin(\theta)\cos(\varphi), \; \; y = r \sin(\theta)\sin(\varphi), \; \; z = r\cos(\theta)
                    \end{equation}
                    The position vector $\vec{r}$ can be written as:
                    \begin{equation}
                        \label{eq:equation21}
                        \vec{r} = r \sin(\theta)\cos(\varphi)\hat{x} + r\sin(\theta)\sin(\varphi)\hat{y} + r \cos(\theta)\hat{z}
                    \end{equation}
                    The scale factors are:
                    \begin{subequations}
                        \begin{align}
                            h_r &= 1\\
                            h_{\theta} &= r\\
                            h_{\varphi} &= r \sin(\theta)
                        \end{align}
                        \flushleft The basis vectors are:
                        \begin{align}
                            \hat{e_r} &= \hat{r}\\
                            \hat{e_\theta} &= \hat{\theta}\\
                            \hat{e_\varphi} &= \hat{\varphi}
                        \end{align}
                        \paragraph{Differential operators:}
                        \begin{subequations}
                            \begin{align}
                                \nabla f  &= \frac{\partial f}{\partial r} \hat{e_r} + \frac{1}{r}\frac{\partial f}{\partial \theta} \hat{e_{\theta}}
                                + \frac{1}{r\sin(\theta)}\frac{\partial f}{\partial \varphi} \hat{e_\varphi}\\
                                \nabla \cdot \vec{F} &= \frac{1}{r^2}\frac{\partial}{\partial r}(r^2 F_r)
                                + \frac{1}{r \sin(\theta)}\frac{\partial}{\partial \theta}(\sin(\theta) F_\theta) + \frac{1}{r \sin (\theta)}\frac{\partial F_\varphi}{\partial \varphi}\\
                                \nabla \times \vec{F} &= \frac{1}{r^2 \sin(\theta)}
                                \begin{vmatrix}
                                    \hat{e_r} && r \hat{e_\theta} && r\sin(\theta)\hat{e_\varphi} \\
                                    \frac{\partial}{\partial r} && \frac{\partial}{\partial \theta} && \frac{\partial}{\partial \varphi} \\
                                    F_r && r F_{\theta} && r\sin(\theta) F_{\theta}
                                \end{vmatrix}\\
                                \nabla^2 f &= \frac{1}{r^2}\frac{\partial}{\partial r}\left(
                                r^2 \frac{\partial f}{\partial r}\right) + \frac{1}{r^2 \sin(\theta)}\frac{\partial }{\partial \theta}\left( \sin(\theta) \frac{\partial f}{\partial \theta} \right)
                                + \frac{1}{r^2 \sin^2(\theta)}\frac{\partial^2 f}{\partial \varphi^2}
                            \end{align}
                        \end{subequations}
                    \end{subequations}
\end{document}